%% ---------------------------------------------------------------------------
%% This file is part of the "Boost Python" talk.
%%
%% Copyright 2015 by Eugen Wintersberger <eugen.wintersberger@gmail.com>
%%
%% This work is licensed under the Creative Commons
%% Attribution-NonCommercial-ShareAlike 4.0 International License. To view 
%% a copy of this license, visit 
%% http://creativecommons.org/licenses/by-nc-sa/4.0/.
%% ---------------------------------------------------------------------------

\begin{frame}[fragile]{The Problem}
    In C++ we may have a function like this 
    \begin{minted}{cpp}
double mysin(double t,double amplitude=1.0,
             double omega=1.0)
{
    return amplitude*std::sin(omega*t);
}
    \end{minted}
    \vspace{0.05\textheight}
    In Python we would like to have something like 
    \begin{minted}{python}
mysin(0.5)
mysin(0.5,omega=1.2)
mysin(0.5,omega=3.0,amplitude=4.5)
    \end{minted}
    \vspace{0.025\textheight}
    Python uses keyword arguments to pass optional arguments, we do not have
    this feature in C++!
\end{frame}

%%%===========================================================================
\begin{frame}[fragile]{\texttt{boost::python::arg} does the trick}
    \begin{minted}{cpp}
#include <boost/python.hpp>

using namespace boost::python; 

double mysin(double t,double amplitude,
             double omega=1.0);

BOOST_PYTHON_MODULE(functions)
{

    def("mysin",mysin,("t",arg("amplitude")=1.0,
                           arg("omega")=1.0));
}


    \end{minted}
\end{frame}

%%%===========================================================================
\begin{frame}{\texttt{boost::python::arg} allows for ...}
    \begin{itemize}
        \item handling default arguments as keyword arguments
        \item introduce new keyword arguments
    \end{itemize}
    \vspace{0.1\textheight}
    Restrictions
    \begin{itemize}
        \item the number of arguments between \texttt{()} must match that of
            the original function
        \item the order of the arguments must match that of the original
            function
    \end{itemize}
\end{frame}
