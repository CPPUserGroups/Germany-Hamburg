%% ---------------------------------------------------------------------------
%% This file is part of the "Boost Python" talk.
%%
%% Copyright 2016 by Eugen Wintersberger <eugen.wintersberger@gmail.com>
%%
%% This work is licensed under the Creative Commons
%% Attribution-NonCommercial-ShareAlike 4.0 International License. To view 
%% a copy of this license, visit 
%% http://creativecommons.org/licenses/by-nc-sa/4.0/.
%% ---------------------------------------------------------------------------

%%%===========================================================================
\begin{frame}[fragile]{Motivation}

    \begin{columns}[t]
        \begin{column}{0.49\linewidth}
            Consider the following class 
            \inputminted[fontsize=\tiny,firstline=26,lastline=36]{cpp}{../src/libtalk/include/talk/device.hpp}
        \end{column}
        \begin{column}{0.49\linewidth}
    And a naiv wrapper
    \begin{minted}[fontsize=\tiny]{cpp}
class_<talk::device>("Device",no_init)
    .def("get_value",&talk::device::get_value)
    .def("set_value",&talk::device::set_value)
    .def("get_sensor",&talk::device::get_sensor)
    .add_property("value",&talk::device::get_value,
                      &talk::device::set_value);
    \end{minted}
        \end{column}
    \end{columns}
    \vspace{0.1\textheight}
    This will cause troubles
    \begin{minted}[fontsize=\small]{python}
>>> d = make_device(Sensor(0.23))
>>> s = d.get_sensor()
>>> del d
>>> s.value #this will fail
    \end{minted}
\end{frame}

%%%===========================================================================
\begin{frame}[fragile]{Solution: call policies}
    \inputminted[fontsize=\small,firstline=59,lastline=65]{cpp}{../src/python-talk/src/sensor.cpp}

    \vspace{0.05\textheight}
    We are using here: \texttt{return\_internal\_reference<>()}
    \begin{itemize}
        \item tells \texttt{boost::python} that the function returns an
            internal reference of an argument
        \item argument index is the first template parameter (default is $1$)
        \item in the case of a class member function the first argument is the
            class instance
    \end{itemize}
\end{frame}
