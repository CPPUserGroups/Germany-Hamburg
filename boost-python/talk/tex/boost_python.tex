
%%%---------------------------------------------------------------------------
\begin{frame}[fragile]{The most basic extension source file}
Need an equivalent to the init-function in Pythons C-API.

\inputminted[fontsize=\tiny,linenos,
             firstline=20,
             lastline=42]{cpp}{../src/python-talk/src/functions.cpp}

\end{frame}

%%%---------------------------------------------------------------------------
\begin{frame}[fragile]{Building \texttt{boost::python} package}
    Directory layout of a project
    \begin{center}
        \begin{tabular}{ll}
            \texttt{setup.py} & script to finally run the build \\
            \texttt{src/}     & directory with C++ source files \\
            \texttt{test/}    & Python unit tests \\
            \texttt{talk/}    & directory with Python code of the package \\
        \end{tabular}
    \end{center}
    \vspace{0.05\textheight}
    \texttt{boost::python} modules are built like any other non-native 
    Python package with C extensions by means of \texttt{setup.py}
    \vspace{0.05\textheight}
    \begin{minted}{bash}
$ python setup.py install --user 
    \end{minted}
    or 
    \begin{minted}{bash}
$ python3 setup.py install --user
    \end{minted}
\end{frame}

%%%---------------------------------------------------------------------------
\begin{frame}[fragile]{The \texttt{setup.py} file}
    \inputminted[fontsize=\tiny,linenos]{python}
    {../src/python-talk/setup.py}
\end{frame}

%%%---------------------------------------------------------------------------
\begin{frame}[fragile]{C++ functions to Python}
    \begin{center}
        \begin{tikzpicture}
            \node (filename) {\texttt{talk/functions.hpp}};
            \node[draw,black,rectangle,below = 0.2cm of filename]
            (header file)
            {
                \begin{minipage}{0.4\linewidth}
    \inputminted[fontsize=\tiny,firstline=23,firstnumber=23]{cpp}
    {../src/libtalk/include/talk/functions.hpp}
    \end{minipage}
            };

            \node[right = 1cm of filename]  (wrap simple) {wrapping unique function};
            \node[draw,rectangle,below = 0.2cm of wrap simple]
            {
                \begin{minipage}{0.35\linewidth}
                    \begin{minted}[fontsize=\tiny]{cpp}
#include <boost/python.hpp>
#include <talk/functions.hpp>

using namespace boost::python;

BOOST_PYTHON_MODULE(functions)
{
    def("div",talk::div);
}
                    \end{minted}
                \end{minipage}
            };

            \node[below = 1cm of header file] (wrap overload) {wrapping overloaded functions};
            \node[rectangle,draw,below = 0.2cm of wrap overload]
            {
                \begin{minipage}{0.7\linewidth}
                    \begin{minted}[fontsize=\tiny]{cpp}
#include <boost/python.hpp>
#include <talk/functions.hpp>

using namespace boost::python;

BOOST_PYTHON_FUNCTION_OVERLOADS(add_overloads,talk::add,2,2);

BOOST_PYTHON_MODULE(functions)
{
    def("add",(double (*)(double,double))2,add_overloads());
    def("add",(int (*)(int,int))2,add_overloads());
}
                    \end{minted}
                \end{minipage}
            };
        \end{tikzpicture}
    \end{center}
\end{frame}

%%%---------------------------------------------------------------------------
\begin{frame}[fragile]{Mapping of primitive data types}
    \begin{center}
        {
            \renewcommand{\arraystretch}{1.5}
            \begin{tabular}{p{0.3\linewidth}|p{0.3\linewidth}}
            \textbf{C++ type} & \textbf{Python type} \\
            \hline
            all integer types & \texttt{int} or \texttt{long} \\
            \texttt{double} & \texttt{float} \\
            \texttt{std::string} & Python string type  \\
        \end{tabular}
    }
    \end{center}
    \vspace{0.1\textheight}
    For all other types converters are required!
\end{frame}

%%%---------------------------------------------------------------------------
\begin{frame}[fragile]{Handling exceptions}
    Two possibilities
    \vspace{0.04\textheight}
    \begin{enumerate}
        \setlength{\itemsep}{0.04\textheight}
        \item translate the C++ exception into an existing Python exception
        \item create a new Python exception in the extension modules scope
    \end{enumerate}
    \vspace{0.1\textheight}
    \begin{center}
        \textbf{The first approach is recommended and should be used whenever
        possible!}
    \end{center}

\end{frame}

%%%---------------------------------------------------------------------------
\begin{frame}[fragile]{Handling exceptions - translation}
    A rather straight forward approach
    \vspace{0.04\textheight}
    \begin{minted}[fontsize=\tiny]{cpp}
#include <boost/python.hpp>
#include <talk/functions.hpp>
#include <talk/exceptions.hpp>

using namespace boost::python;

void division_by_zero_translator(const talk::division_by_zero &)
{
    PyErr_SetString(PyExc_ZeroDivisionError,"only Chuck Norris can do this!");
}

BOOST_PYTHON_MODULE(functions)
{
    def("div",talk::div);
    register_exception_translator<talk::division_by_zero>(division_by_zero_translator);
}
    \end{minted}
\end{frame}

%%%---------------------------------------------------------------------------
\begin{frame}[fragile]{Handling exceptions - new exception}

    \begin{minted}[fontsize=\tiny]{cpp}
#include <boost/python.hpp>
#include <talk/functions.hpp>
#include <talk/exceptions.hpp>

using namespace boost::python;

static object TalkError;
static char *TalkError_Doc = "Internal error in talk library";

void talk_error_translator(const talk::talk_error &)
{
    PyErr_SetString(TalkError.ptr(),"An internal error");
}

//=================implementation of the python extension======================
BOOST_PYTHON_MODULE(functions)
{
    TalkError = object(handle<>(PyErr_NewExceptionWithDoc("talk.functions.TalkError",TalkError_Doc,
                                                          nullptr,nullptr)));
    scope().attr("TalkError")=TalkError;

    register_exception_translator<talk::talk_error>(talk_error_translator);
   
}
    \end{minted}
\end{frame}

%%%---------------------------------------------------------------------------
\begin{frame}[fragile]{C++ classes to Python}
    \todo[inline]{how to wrap classes}
\end{frame}

%%%---------------------------------------------------------------------------
\begin{frame}[fragile]{Dealing with constructors}
    \todo[inline]{describe how to make constructors available}
\end{frame}

%%%---------------------------------------------------------------------------
\begin{frame}[fragile]{Dealing with inheritance}
    \todo[inline]{how to manage inheritance}
\end{frame}

%%%---------------------------------------------------------------------------
\begin{frame}[fragile]{Python properties}
    \todo[inline]{how to do properties}
\end{frame}



